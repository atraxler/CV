\documentclass[10pt]{article}
\usepackage{array, xcolor, bibentry}
\usepackage[margin=3cm]{geometry}
\usepackage{longtable}
 
\title{\bfseries\Large Adrienne L. Traxler}
\author{adrienne.traxler@wright.edu}
\date{}
 
\definecolor{lightgray}{gray}{0.8}
\newcolumntype{L}{>{\raggedleft}p{0.14\textwidth}}
\newcolumntype{R}{p{0.8\textwidth}}
\newcommand\VRule{\color{lightgray}\vrule width 0.5pt}

% Too much white space in bullet lists
\newcommand{\squishlist}{
	\begin{list}{$\bullet$}
		{ \setlength{\itemsep}{0pt}
			\setlength{\parsep}{3pt}
			\setlength{\topsep}{3pt}
			\setlength{\partopsep}{0pt}
			%\setlength{\leftmargin}{1.5em}
			\setlength{\labelwidth}{1em}
			\setlength{\labelsep}{0.5em} } }
	\newcommand{\squishend}{
\end{list}  }
 
 
\begin{document}
\maketitle
\vspace{1em}
\begin{minipage}[ht]{0.48\textwidth}
Department of Physics\\
Wright State University\\
Dayton, OH 45435
\end{minipage}
\begin{minipage}[ht]{0.48\textwidth}
Office: 251 Fawcett Hall \\
Phone: 937-985-3140 \\
\url{https://github.com/atraxler}
\end{minipage}
\vspace{20pt}

\section*{Education}
\begin{tabular}{L!{\VRule}R}
	2006--2011&{\bf PhD in Applied Mathematics and Statistics}, University of California Santa Cruz, Santa Cruz, CA.\\[5pt]
	2004--2006&{\bf MS in Teaching (physics concentration)}, University of Maine, Orono, ME.\\
	2000--2004&{\bf BS in Physics}, University of Maine, Orono, ME.\\
\end{tabular}


 
\section*{Professional Experience}
\begin{tabular}{L!{\VRule}R}
2019--present & {\bf Associate Professor of Physics}, Wright State University, Dayton, OH.\\
2014--2019 & {\bf Assistant Professor of Physics}, Wright State University, Dayton, OH.\\
2011--2014 & {\bf Assistant Director of Research Programs}, Department of Physics, Florida International University, Miami, FL.\\
March--July 2011 & {\bf Postdoctoral Researcher}, Department of Applied Mathematics and Statistics, University of California Santa Cruz, Santa Cruz, CA.\\
\end{tabular}
 
% Prints blank on first compiling. What works: Change \nobibliography
% to \bibliography below, recompile, then return to \nobibliography
% and repeat. 
\bibliographystyle{plain}
\nobibliography{my_pubs}

\renewcommand{\arraystretch}{1.5}

\section*{Publications} 

\subsection*{Journal Articles}
%\begin{tabular}{L!{\VRule}R}
\begin{longtable}{L!{\VRule}R}
2022 	& \bibentry{wolf_social_2022}\\
		& \bibentry{traxler_networks_2022}\\
		& \bibentry{commeford_characterizing_2022}\\		
2021	& \bibentry{commeford_characterizing_2021}\\
		& \bibentry{brewe_transitioning_2021}\\
2020 	& \bibentry{traxler_network_2020}\\
		& \bibentry{traxler_sex_2020}\\
		& \bibentry{wells_2020_exploring}\\
2019 	& \bibentry{wells_2019_exploring}\\
		& \bibentry{blue_resource_2019}\\
		& \bibentry{henderson_partitioning_2019}\\[5pt]
2018 	& \bibentry{traxler_networks_2018}\\[5pt]
 		& \bibentry{henderson_item-level_2018}\\[5pt]
 		& \bibentry{blue_gender_2018}\\[5pt]
 		& \bibentry{traxler_gender_2018}\\[5pt]
2017 	& \bibentry{henderson_exploring_2017}\\[5pt]
 		& \bibentry{schroeder_humanizing_2017}\\[5pt]
2016 	& \bibentry{traxler_enriching_2016}\\[5pt]
2015 	& \bibentry{traxler_equity_2015}\\[5pt]
2014 	& \bibentry{artze-vega_stereotype_2014}\\[5pt]
2013 	& \bibentry{brewe_extending_2013}\\[5pt]
2012 	& \bibentry{mirouh2012new}\\[5pt]
2011 	& \bibentry{rosenblum2011turbulent}\\[5pt]
		& \bibentry{traxler2011numerically}\\[5pt]
		& \bibentry{stellmach2011dynamics}\\[5pt]
		& \bibentry{traxler2011dynamics}\\%[5pt]
%\end{tabular}
\end{longtable}

\subsection*{Book Chapters}
\begin{tabular}{L!{\VRule}R}
	%\begin{longtable}{L!{\VRule}R}
2020	& \bibentry{traxler_disability_2020}\\
	& \bibentry{traxler_who_2020}\\
	%\end{longtable}
\end{tabular}

\subsection*{Work Under Review}
\begin{tabular}{L!{\VRule}R}
	%\begin{longtable}{L!{\VRule}R}
	& \bibentry{barthelemy_gender}\\
	%\end{longtable}
\end{tabular}

\subsection*{Work in Preparation}
\begin{tabular}{L!{\VRule}R}
	%\begin{longtable}{L!{\VRule}R}
	& \bibentry{jesper_classify}\\
	%\end{longtable}
\end{tabular}

%\newpage

\subsection*{Conference Proceedings}
%\begin{tabular}{L!{\VRule}R}
\begin{longtable}{L!{\VRule}R}
2019	& \bibentry{myers_quantifying_2019}\\[5pt]
		& \bibentry{commeford_characterizing_2019}\\[5pt]
2018	& \bibentry{myers_content_2018}\\[5pt]
2016 	& \bibentry{traxler_coursenetworking_2016}\\[5pt]
		& \bibentry{sandt_TNT_2016}\\[5pt]
2015 	& \bibentry{traxler_community_2015}\\[5pt]
		& \bibentry{traxler_multiple_2015}\\[5pt]
2014 	& \bibentry{mahadeo_epistemic_2014}\\[5pt]
2010 	& \bibentry{traxler_fluid_2010}\\[5pt]
2006 	& \bibentry{traxler_students_2006}\\%[5pt]
\end{longtable}

\subsection*{Other Publications}
\begin{tabular}{L!{\VRule}R}
2011	& \bibentry{traxler_PhD_2011}\\[5pt]
2010 	& \bibentry{traxler_linear_2009}\\[5pt]
2006 	& \bibentry{traxler_assessment_2006}\\%[5pt]
\end{tabular}
 
\section*{Funded Research}

\paragraph{National Science Foundation:} DUE-2100024. Collaborative Research: Mapping professional support networks of women and gender and sexual minorities in physics. July 2021--June 2024. Budget: \$498,733 total, \$63,481 to Wright State University. PI (collaborative proposal with Ram\'{o}n Barthelemy, University of Utah, and Charles Henderson, Western Michigan University).

\paragraph{National Science Foundation:} DUE-2027963. RAPID: Collaborative Research: Faculty Networks Supporting Rapid Transitions to Online Physics Teaching During the COVID-19 Pandemic. May 2020--April 2021. Budget: \$86,978 total, \$15,083 to Wright State University. PI (collaborative proposal with Eric Brewe, Drexel University).

\paragraph{National Science Foundation:} DUE-1742339. WSU Students ASK$\ldots$ A Success \& Scholarship Program for Students Applying Scientific Knowledge. August 2017--August 2022. Budget: \$997,589. Co-PI (with Jason Deibel, PI, and Meredith Rodgers, co-PI).

\paragraph{National Science Foundation:} DUE-1712341. Collaborative Research: Characterizing Active Learning Environments in Physics. July 2017--June 2020. Budget: \$299,981 total, \$73,485 to Wright State University. PI (collaborative proposal with Eric Brewe, Drexel University).

\paragraph{National Science Foundation:} Subaward of HRD-1436702. Implementation Project: Improving Pathways for STEM Retention and Graduation. August 2015--July 2016. Budget: \$56,011. PI at Wright State University (co-PIs Noah Schroeder, Douglas Petkie), subcontract from Central State University.

 
%\newpage
 
\section*{Invited Talks and Workshops}
%\begin{tabular}{L!{\VRule}R}
\begin{longtable}{L!{\VRule}R}
2021 	& \bibentry{traxler_WAAPT_2021}\\
2020 	& \bibentry{traxler_SOSAAPT_2020}\\
		& \bibentry{blue_AAPTtalk_2020}\\
		& \bibentry{traxler_MSU_2020}\\
2019	& \bibentry{traxler_CONCORtalk_2019}\\
		& \bibentry{traxler_PERC_2019}\\[5pt]
		& \bibentry{traxler_JSU_2019b}\\[5pt]
		& \bibentry{traxler_JSU_2019a}\\[5pt]
2018	& \bibentry{traxler_equitytalk_2018}\\[5pt]
		& \bibentry{traxler_gendertalk_2018}\\[5pt]
		& \bibentry{traxler_NWtalk_2018}\\%[5pt]
		& \bibentry{traxler_OSAPS_2018}\\%[5pt]
2017	& \bibentry{traxler_ESERA_2017}\\%[5pt]
		& \bibentry{traxler_PERCpanel_2017}\\%[5pt]
		& \bibentry{traxler_AAPT_2017}\\%[5pt]
		& \bibentry{traxler_PERCoGS_2017}\\%[5pt]
		& \bibentry{traxler_APS_2017}\\%[5pt]
2016	& \bibentry{traxler_OSU_2016}\\%[5pt]
2015	& \bibentry{traxler_Miami_2015}\\%[5pt]
		& \bibentry{traxler_Purdue_2015}\\%[5pt]
		& \bibentry{traxler_AAPT_2015}\\%[5pt]
		& \bibentry{traxler_MSU_2015}\\%[5pt]
2014	& \bibentry{traxler_WSU_2014}\\%[5pt]
		& \bibentry{traxler_WAAPT_2014}\\
2013	& \bibentry{traxler_Purdue_2013}\\
		& \bibentry{traxler_global_2013}\\
		& \bibentry{traxler_OSU_2013}\\
		& \bibentry{traxler_AAPT_2013}\\
2012	& \bibentry{traxler_FIU_2012}\\
2011	& \bibentry{traxler_Sonoma_2011}\\

\end{longtable}

\section*{Contributed Talks and Posters}
\begin{longtable}{L!{\VRule}R}
2021	& \bibentry{traxler_APS_2021}\\
		& \bibentry{brewe_AAPT_2021}\\
	 	& \bibentry{mellen_AAPT_2021}\\
		& \bibentry{green_AAPT_2021}\\
2020 	& \bibentry{traxler_PERCposter_2020}\\
		& \bibentry{traxler_CALEPposter_2020}\\
		& \bibentry{traxler_SSTEMtalk_2020}\\
		& \bibentry{commeford_AAPTtalk_2020}\\
2019	& \bibentry{traxler_PERCposter_2019}\\[5pt]
		& \bibentry{myers_PERCposter_2019}\\[5pt]
		& \bibentry{suda_AAPTposter_2019}\\[5pt]
		& \bibentry{commeford_AAPTtalk_2019}\\[5pt]
		& \bibentry{wells_AAPT_2019}\\[5pt]
		& \bibentry{deibel_CUR_2019}\\[5pt]
		& \bibentry{deibel_PKAL_2019}\\[5pt]
		& \bibentry{commeford_APS_2019}\\[5pt]
		& \bibentry{eledkawi_LSAMP_2019}\\[5pt]
		& \bibentry{myers_CogFest_2019}\\[5pt]
2018	& \bibentry{myers_PERCposter_2018}\\[5pt]
		& \bibentry{commeford_PERC_2018}\\[5pt]
		& \bibentry{henderson_AAPT_2018}\\[5pt]
2017	& \bibentry{commeford_PERC_2017}\\%[5pt]
		& \bibentry{rundquist_AAPT_2017}\\%[5pt]
		& \bibentry{henderson_AAPT_2017}\\%[5pt]
		& \bibentry{traxler_APSposter_2017}\\%[5pt]
		& \bibentry{lindell_APS_2017}\\%[5pt]
2016	& \bibentry{traxler_PERCposter_2016}\\%[5pt]
		& \bibentry{bruun_PERCposter_2016}\\
		& \bibentry{hierath_PERCposter_2016}\\
		& \bibentry{papak_PERCposter_2016}\\
		& \bibentry{sandt_PERCposter_2016}\\
		& \bibentry{traxler_AAPT_2016}\\
		& \bibentry{papak_WAAPT_2016}\\
2015	& \bibentry{traxler_WSUTSS_2015}\\
		& \bibentry{traxler_PERCposter_2015}\\
		& \bibentry{hierath_WSU_2015}\\
		& \bibentry{blue_Miami_2015}\\
2014	& \bibentry{traxler_PERCposter_2014b}\\
		& \bibentry{traxler_PERCposter_2014a}\\
		& \bibentry{manthey_multi-measure_2014}\\
		& \bibentry{manthey_general_2014}\\
2013	& \bibentry{traxler_PERCposter_2013}\\
		& \bibentry{mahadeo_PERCposter_2013}\\
		& \bibentry{traxler_AAPTposter_2013}\\
		& \bibentry{mahadeo_AAPTtalk_2013}\\
		& \bibentry{kavallieratos_ACStalk_2013}\\
2012	& \bibentry{frazier_PERCposter_2012}\\
		& \bibentry{traxler_TRUSEposter_2012}\\
2011	& \bibentry{traxler_thermohaline_2011}\\
2009	& \bibentry{traxler_spontaneous_2009}\\
2008	& \bibentry{traxler_spontaneous_2008}\\
		& \bibentry{stellmach_three-dimensional_2008}\\
2006	& \bibentry{traxler_analysis_2006}\\
		& \bibentry{traxler_refining_2006}\\
2005	& \bibentry{traxler_student_2005}
\end{longtable}

\section*{Teaching Experience}

\subsection*{Wright State University}

%\paragraph{
%\begin{itemize}
\squishlist
\item \textbf{Undergraduate Physics Seminar I (PHY 1000, 1 credit.)} (F2018, F2019, F2020, F2021) Introductory undergraduate seminar, 5--10 students. %First-year seminar taken by physics majors to orient them to the department, build community, and develop laboratory skills. 

\item \textbf{General Physics I (PHY 2400, 4 credits).} (F2014, Su2015, F2015, Sp2016, F2016, F2017, F2018, F2019, Sp2020, F2020) Introductory undergraduate, 10--220 students. Calculus-based course covering Newtonian mechanics. %taken by physics and engineering majors covering Newtonian mechanics (kinematics, forces, energy, momentum, rotation, oscillations). 

\item \textbf{Introduction to Astrophysics (PHY 3300, 3 credits).} (Sp2018, Sp2019, Sp2021) Upper-division undergraduate, 5--15 students. Developed new astrophysics course. 

\item \textbf{Analytical Mechanics (PHY 3710/5710, 3 credits).}
(Sp2017, Sp2018, Sp2019, Sp2020, Sp2022) Upper-division undergraduate and cross-listed as graduate, 10--20 students. %Course for physics and applied mathematics majors covering
Newtonian mechanics using ordinary differential equations, with computational assignments in Matlab.

\item \textbf{Senior Project (PHY 4940, 3 credits).} 
(F2017, Sp2018, F2018, Sp2019) Capstone undergraduate independent study format, 1--2 students. 

\item \textbf{Classical Mechanics (PHY 6800, 3 credits).} 
(F2021) Graduate, lecture format, 6 students. Mechanics including Lagrangian formulation, orbital dynamics, and oscillations.

\item \textbf{Special Topics in Physical Science for Teachers (PHY 6990, 3 credits).} (F2017, F2018, F2020) Graduate, online format, 1--2 students. Developed new course for high school teachers seeking certification to teach college-credit physics courses. %Mathematical and conceptual review of introductory course content, paired with 
Pedagogical readings, writing, and discussion around introductory course content. 

\item \textbf{Minor Problems (PHY 7990, 3 credit hours).}
(F2014, Sp2015) Graduate, independent study format, 1--2 students. Research credits taken by graduate students before formally submitting thesis proposal. 

\item \textbf{Research (PHY 8990, 3 credit hours).} 
(F2015, Sp2016, Su2016, F2017, Sp2017, Su2017, F2017, Sp2018, F2020, Sp2021) Graduate, independent study format, 1--3 students. Thesis project credits.
%\end{itemize}
\squishend

\subsection*{Florida International University}

\squishlist
%\begin{itemize}
\item \textbf{Calculus-based Physics I (PHY 2048, 4 credit hours).} 
(F2012) Introductory undergraduate lecture and laboratory (studio format), 25--30 students. Calculus-based. %course taken by physics and other science majors, covering Newtonian mechanics. 

\item \textbf{Calculus-based Physics II (PHY 2049, 4 credit hours).} (Sp2013)
Introductory undergraduate lecture and laboratory (studio format), 25--30 students. Second semester of Modeling Instruction course above (PHY 2048), covering electricity and magnetism. 
%\end{itemize}
\squishend


\section*{Student Supervision}

\subsection*{Graduate Students}
\squishlist
\item Hannah Benston (MS physics 2022)
\item Kelley Commeford (co-advisor, PhD physics 2021), ``Characterizing active learning environments in physics''
\item Raym Alzahrani (MS physics 2016)
\item Sarah Hierath (MS physics 2016)
\item Emily Sandt (MS physics 2016)
\squishend

\subsection*{Undergraduate Students}
\squishlist
\item Heather Robinson (Senior project, BS expected 2022)
\item Chynna Spitler (Senior project co-advisor, BS expected 2022)
\item Carissa Myers (Senior project, BS 2019)
\item Tyme Suda (Senior project, BS 2019)
\item Hannah Roth (Senior project, BS 2018)
\item Jonathan Mahadeo (Research assistant, 2012--2014)
\item Amber Frazier (PhysTEC REU student, 2012)
\squishend

\subsection*{Thesis and Project Committees}
\squishlist
\item Current: Hope Strickland (EdD), Justin Gambrell (PhD physics)
\item Past: Colin Green (MS physics, 2022), Russell Clark (MST, 2016), Melody Deitrick (MST, 2020)
\squishend


\section*{Leadership Positions}

\squishlist
\item \textbf{PERLOC} (Physics Education Research Leadership and Organizing Council); Member, January 2020--August 2021, Vice-chair, January--August 2021

\item \textbf{Southern Ohio Section of the American Association of Physics Teachers} Vice-president for four-year colleges (Spring 2018--Spring 2019), treasurer (Spring 2019--present)

\item \textbf{American Physical Society} Forum on Education Member-at-Large, 2019--2021.

\item \textbf{Physics Education Research Conference Proceedings} Editor, Spring 2016--Spring 2019

\item \textbf{American Association of Physics Teachers} Committee on Women in Physics (member, 2015; vice chair, 2016; chair, 2017)

\item \textbf{Faculty Online Learning Community} Peer Leader for APS/AAS/AAPT Physics and Astronomy New Faculty Workshop (Spring 2016--Fall 2017)
\squishend


\section*{Other Service}

\subsection*{Workshop Instructor (The Carpentries)}

The Carpentries workshops (Software Carpentry, Data Carpentry, and Library Carpentry) teach programming and data analysis tools to academic audiences. 

\squishlist
\item Library Carpentry: Virginia Tech (June 2021, lead instructor)
\item Data Carpentry: North Carolina Central University (August 2020, assistant instructor) and US Department of Agriculture (May 2021, assistant instructor).
\squishend


\subsection*{Review Work}

\squishlist
\item \textbf{Journals:} American Journal of Physics, CBE – Life Sciences Education, European Journal of Physics, International Journal of Research and Method in Education, Physical Review Physics Education Research, The Physics Teacher, Physics Education Research Conference Proceedings, Proceedings A, Science Advances
\item \textbf{Grants:} National Science Foundation
\squishend

 
\end{document}